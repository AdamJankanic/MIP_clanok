% Metódy inžinierskej práce

\documentclass[10pt,twoside,slovak,a4paper]{article}

\usepackage[slovak]{babel}
%\usepackage[T1]{fontenc}
\usepackage[IL2]{fontenc} % lepšia sadzba písmena Ľ než v T1
\usepackage[utf8]{inputenc}
\usepackage{graphicx}
\usepackage{url} % príkaz \url na formátovanie URL
\usepackage{hyperref} % odkazy v texte budú aktívne (pri niektorých triedach dokumentov spôsobuje posun textu)

\usepackage{cite}
%\usepackage{times}

\pagestyle{headings}

\title{Použitie vrituálnych laboratórií pri výučbe chémie \\
 
 {\normalsize Semestrálny projekt v predmete Metódy inžinierskej práce, ak. rok 2020/21\\
 \normalsize vedenie: Jozef  Sitarčík}} % meno a priezvisko vyučujúceho na cvičeniach

\author{Adam Jankanič\\[2pt]
	{\small Slovenská technická univerzita v Bratislave}\\
	{\small Fakulta informatiky a informačných technológií}\\
	{\small \texttt{xjankanic@stuba.sk}}
	}

\date{\small 30. september 2020} % upravte



\begin{document}

\maketitle

\begin{abstract}
V tejto práci sledujeme použitie virtuálnych laboratórií, miesto klasických metód vzdelávania. 
Virtuálne laboratória sú skvelý prostriedok pre výučbu v mestách, alebo aj krajinách, kde je málo laboratórií. 
Tieto vzdialené laboratória dokážu zabezpečiť adekvátne vzdelanie pre študentov bez rozdielu od vzdialenosti. 
Používajú sa napríklad pri výučbe chémie, kde je najlepší spôsob získania vedomostí a chápania práve pozorovanie a robenie pokusov. 
V tomto dokumente si zhrnieme výskumy, niekoľko pozorovaní študentov a ich dosiahnuté výsledky pri výučbe chémie vo virtuálnych laboratóriách.
\end{abstract}



\section{Úvod}

Výučba v školách je náročný proces, o to viac, keď sa jedná o zložitý predmet ako chémia. Nejedná sa iba o teoretické vedomosti, ale dôležité sú aj praktické zručnosti. Kvôli týmto veciam považuje množstvo študentov štúdium ako neatraktívne a vyberú si iný smer. Hlavným prostriedkom pri výučbe sú laboratória, ale nie sú samozrejmosťou všade na svete. Vybavenie v nich je pre mnohé školy finančne náročné a úroveň vzdelania klesá. To ale nie je jediný problém. Sú krajiny, kde je veľký nepomer študentov a učiteľov, teda vzniká situácia, kedy musí veľkú skupinku študentov učiť jeden učiteľ. Pri takejto výučbe nie je možné odpovedať na otázky všetkých študentov, zabezpečiť všetkým praktické precvičenie v laboratórií, či  ukážku pokusu. Dôsledkom toho je degradácia výučby, produkovanie menšieho počtu kvalitných chemikov a tým pádom aj spomalenie výskumov.


\section{Dôležitosť laboratórií} \label{d_laboratorii}
Koncept ''learning by doing'' je všeobecne uznávaný spôsob výučby. Výučba touto metódou nie je len o teoretických vedomostiach, ale hlavne o praktických skúsenostiach. Tie sú veľmi dôležité a najmä v chémii, kde je potrebné, aby študenti jasne rozumeli, čo sa od nich žiada. Touto metódou si môžu vyskúšať svoje znalosti v praxi a sú nepriamo nútení viac uvažovať nad tým, čo robia. V laboratóriu sa stávajú aktívnymi a učia sledovaním, zisťovaním a najmä robením. Študenti sa takto dokážu učiť aj zo svojich chýb, ktoré ich posunú ďalej viac, akoby len nečinne prihliadali. To ma v konečnom dôsledku pozitívny vplyv pri ich výuke a zároveň podporuje ich samostatnosť. Hoci laboratória sú pri výučbe dôležité, nie každá škola má na to finančné prostriedky. V takomto prípade je výučba presný opak ''learning by doing''. Študenti sú obraní o možnosť nadobudnutia praktických skúseností. Preto sa začali hľadať alternatívy, ako za použitia technológií nahradiť klasickú výučbu v laboratóriu.

\section {Virtuálne laboratória} \label{virtualne_laboratoria}
Virtuále laboratória sú simulátory, ktoré poskytujú realistické prevedenie klasických laboratórií vo virtuálnej forme. Zabezpečujú študetom laboratórne skúsenosti bez toho, aby museli fyzicky byť v laboratóriu. Môžu to byť jednoduché videá, ale aj 3D interaktívne prostredie. Laboratórne vybavenie a prostriedky majú takmer identické správanie ako v realite. Vďaka tomu môžu študenti robiť experimenty a výsledky sú autentické s tými z reálnych laboratórií.


\subsection {Výhody a nevýhody virtualných laboratórií} \label{vyhody_nevyhody}
 Výhody
\begin{enumerate}
\item Finančná úspora - zabezpečenie virtuálneho systemu je cenovo výhodnejšie pre školy, ako kúpa špičkového laboratória. Cena virtuálneho systému pre inštitúciu je 2-20\$ na študenta, pri celoplošnom systéme je to iba 1-5\$ na študenta.
\item Flexibilita - rôzne simulácie experimntov, ktoré vyžadujú rôzne aparatúry môžu byť ľahko a rýchlo vytvárané

\item Dostupnosť vybavenia - v simulácii može niekoľko študentov súčasne využívať rovanké vybavenie, v realite by bolo nutnosťou kúpa ďalšieho vybavenia

\item Bezpečnosť - študenti nie sú priamo v laboratóriu, čiže sú chránení pred potenciálnym nebezpečenstvom

\item Vybavenie je virtuálne - pomôcky nie sú fyzické, takže sa študenti a učitelia nemusia obávať, že sa niečo rozbije, či pokazí

\end{enumerate}
Nevýhody
\begin{enumerate}
    \item Konfigurácia - systémy bývajú zložité a časovo náročné na konfiguráciu
    \item Žiadne následky - študentom to môže pripadať ako hra. Všetko čo robia je len simulácia a nevyplývajú z nej žiadne následky. Študenti sa kvôli tomu môžu byť menej zodpovední, opatrní a nemusia to brať vážne.
    \item Prax - aj keď je simulácia na nerozoznanie od reality, nadobudnutie určitých praktických schopností si vyžaduje skutočné vybavenie a čas strávený v laboratóriu.
\end{enumerate}


\section{Požiadavky} \label{poziadavky}

Základným problémom je teda\ldots{} Najprv sa pozrieme na nejaké vysvetlenie (časť~\ref{ina:nejake}), a potom na ešte nejaké (časť~\ref{ina:nejake}).\footnote{Niekedy môžete potrebovať aj poznámku pod čiarou.}

Môže sa zdať, že problém vlastne nejestvuje\cite{Coplien:MPD}, ale bolo dokázané, že to tak nie je~\cite{Czarnecki:Staged, Czarnecki:Progress}. Napriek tomu, aj dnes na webe narazíme na všelijaké pochybné názory\cite{PLP-Framework}. Dôležité veci možno \emph{zdôrazniť kurzívou}.


\subsection{Nejaké vysvetlenie} \label{ina:nejake}

Niekedy treba uviesť zoznam:

\begin{itemize}
\item jedna vec
\item druhá vec
	\begin{itemize}
	\item x
	\item y
	\end{itemize}
\end{itemize}

Ten istý zoznam, len číslovaný:

\begin{enumerate}
\item jedna vec
\item druhá vec
	\begin{enumerate}
	\item x
	\item y
	\end{enumerate}
\end{enumerate}


\subsection{Ešte nejaké vysvetlenie} \label{ina:este}

\paragraph{Veľmi dôležitá poznámka.}
Niekedy je potrebné nadpisom označiť odsek. Text pokračuje hneď za nadpisom.



\section{Experiment 1 a 2} \label{experiment}




\section{Zhrnutie} \label{zhrnutie}




\section{Záver} \label{zaver} % prípadne iný variant názvu



%\acknowledgement{Ak niekomu chcete poďakovať\ldots}


% týmto sa generuje zoznam literatúry z obsahu súboru literatura.bib podľa toho, na čo sa v článku odkazujete
\bibliography{literatura}
\bibliographystyle{plain} % prípadne alpha, abbrv alebo hociktorý iný
\end{document}
